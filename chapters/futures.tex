% !TeX root = ../main.tex

\chapter{总结与展望}

本文从网格切割问题出发,介绍了网格和网格切割问题的定义,之后分别介绍了几类问题和其部分实现。

第一类问题是给定割线集,如何生成切割后的网格。这个问题可以通过将所有三角面切割为一个个“孤岛”,再通过连接不在割线上来解决。但是,上述算法没有考虑该问题的局域性质。我们在文章中介绍了一个基于 Euler 回路的切割算法。

第二类问题是如何在优化割线长度的意义下,生成可以将网格切割为圆盘拓扑的割线集。\citet{Chai2018} 的工作中发现,计算网格的柄圈可能是这种问题的好的启发。这引导着我们将 \citet{Busaryev2012} 中的方法应用在计算 $ H_1(\mathbb{O}) $ 的一组最短基的问题中,即用 TetGen 构造内部空间的单复形表示,再构造链群,闭链群,边缘链群和同调群等线性空间的基表示,用同调坐标标记每个边,最后利用最短路径树构造 $ \mathcal{M} $ 上的圈集合,再筛选出分居不同 $ H_1(\mathbb{O}) $ 的等价类的、长度最短的圈。

本文的方法仍然存在着很多不足。

\citet{Dey2008} 中的提到的柄圈计算算法,采取用单复形序列逐级构造的办法,只需要进行内部空间的剖分就可以进行计算;之后的文献\cite{Dey2013}则实现了利用 Reeb Graph 来求解近似最短的柄圈集,克服了本文提到的方法需要网格剖分环节的缺陷。本文应该进一步参考研究上述实现。

同时,很多几何处理应用都需要更精细,与具体应用联合优化,或用户可调节的切割方案。未来工作中我也会进一步探索交互式的切割方案生成机制。