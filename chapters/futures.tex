% !TeX root = ../main.tex

\chapter{总结与展望}

本文从网格切割问题出发,分别介绍了给定割缝如何生成切割后的网格,GIM 网格切割算法以及优化割缝长度的、基于柄圈的网格切割算法,同时选取了部分算法进行实现和比较。

本文的方法仍然存在着很多不足。基于柄圈的网格切割算法需要筛选的圈数量为 $ O(|V||E|) $ 级别,这对于较大的网格来说仍然是很大的开销。同时,求取柄圈的过程中需要计算外部边界的剖分,剖分的时间开销也是不可忽略的。

\citet{Dey2008} 中的提到的柄圈计算算法,采取用单复形序列逐级构造的办法,只需要进行内部空间的剖分就可以进行计算;之后的文献\cite{Dey2013}则实现了利用 Reeb Graph 来求解近似最短的柄圈集,克服了本文提到的方法需要网格剖分环节的缺陷。本文应该进一步参考研究上述实现。

同时,很多几何处理应用都需要更精细,与具体应用联合优化,或用户可调节的切割方案。未来工作中我也会进一步探索交互式的切割方案生成机制。