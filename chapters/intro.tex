% !TeX root = ../main.tex

\chapter{绪论}

\section{网格切割}

三维网格的许多几何处理过程,如参数化和纹理映射,都依赖网格的拓扑信息。在某些场景下,我们希望能将带有复杂拓扑的网格,在保持几何信息近似不变的情形下变为圆盘拓扑。

以纹理映射问题为例,我们希望获得一个网格曲面到 $ [0, 1] \times [0, 1] $ 的双射,而此映射存在的必要条件是两个拓扑空间同胚。这时,唯一的方法就是将该网格曲面切割,使结果拓扑同胚于圆盘。

对于不同的几何处理过程,对三维网格的要求各不相同,切割时应加以考虑。对于纹理映射来说,一个常见的要求是寻找使参数化后三角形形变最小的切割方案\cite{Gu2002}。另一个常见的要求是令割线尽可能短,这在提高纹理映射后图集生成的空间利用率上很有帮助\cite{atlasgen}。

\section{相关工作}

由于网格切割在计算机图形学中的重要地位,切割和割线优化问题被研究者们广泛研究。寻找合法的割线是快速的,\citet{Dey1995} 说明只需要对对偶图进行广度优先搜索,就可以在 $ O(n) $ 的时间复杂度下得到将网格切割为圆盘拓扑的割线。

但是,优化割线长度是比较困难的。\citet{Erickson2002} 证明了,对于二维流形网格来说,计算其总长度或总段数意义下的最短割线是 NP-Hard 问题,并且提出了 $ O(g^2 n \log n) $ 时间复杂度的最小切割图近似算法。

\section{本文内容}

本论文将介绍网格切割的基本方法,以及优化切割方案以实现较短割线的几种算法,最后进行比较。

% \subsection{二级节标题}

% \subsubsection{三级节标题}

% \paragraph{四级节标题}

% \subparagraph{五级节标题}

% 本模板 \pkg{ustcthesis} 是中国科学技术大学本科生和研究生学位论文的 \LaTeX{}
% 模板, 按照《\href{http://gradschool.ustc.edu.cn/static/oldsite/ylb/material/xw/wdxz/32.pdf}
% {中国科学技术大学研究生学位论文撰写手册}》(以下简称《撰写手册》)和
% 《\href{https://www.teach.ustc.edu.cn/notice/notice-teaching/11530.html}
% {关于本科毕业论文(设计)格式和统一封面的通知}》的要求编写。

% Lorem ipsum dolor sit amet, consectetur adipiscing elit, sed do eiusmod tempor
% incididunt ut labore et dolore magna aliqua.
% Ut enim ad minim veniam, quis nostrud exercitation ullamco laboris nisi ut
% aliquip ex ea commodo consequat.
% Duis aute irure dolor in reprehenderit in voluptate velit esse cillum dolore eu
% fugiat nulla pariatur.
% Excepteur sint occaecat cupidatat non proident, sunt in culpa qui officia
% deserunt mollit anim id est laborum.



% \section{脚注}

% Lorem ipsum dolor sit amet, consectetur adipiscing elit, sed do eiusmod tempor
% incididunt ut labore et dolore magna aliqua.
% \footnote{Ut enim ad minim veniam, quis nostrud exercitation ullamco laboris
%   nisi ut aliquip ex ea commodo consequat.
%   Duis aute irure dolor in reprehenderit in voluptate velit esse cillum dolore
%   eu fugiat nulla pariatur.}
