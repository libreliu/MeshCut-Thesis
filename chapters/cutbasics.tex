% !TeX root = ../main.tex

\chapter{网格切割的基本方法}

\section{基本概念}

本文研究的网格是指由三角面片组成的(紧)流形网格。对于流形网格,任意一个边相邻的三角面片的数目不超过两个。

网格的割线是指一个边集合,将此边集合中的边去除\footnote{去除两个三角面片的公共边,即将两个三角面片“分开”。}后,剩余的网格构成圆盘拓扑。

\section{如何沿割线展开曲面}



\section{洪泛算法}

\citet{Gu2002} 中提出的网格切割方法基于这样一个简单的事实:三角面片本身是圆盘拓扑。考虑这样的拓扑空间,其由一些三角面片组成,边界为三角面片外的边。

\begin{algorithm}[h]
    \SetAlgoLined
    \KwData{this text}
    \KwResult{how to write algorithm with \LaTeX2e }
  
    移除种子三角形.
    \While{有只和一个三角形 $ t $ 相邻的边 $ e $} {
      read current\;
      \eIf{understand}{
        go to next section\;
        current section becomes this one\;
      }{
        go back to the beginning of current section\;
      }
    }
    \caption{算法示例1}
\end{algorithm}
  

\section{单纯同调理论}

\citet{oncomputinghantun} 基于几何直观,发现二维流形网格的“把手”和“洞”等特征对于拓扑简化和特征检测都很有帮助。下面的篇幅对其用到的代数几何内容做一个简要介绍。
