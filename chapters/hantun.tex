% !TeX root = ../main.tex

\chapter{基于环柄的网格切割算法}

\section{简介}

\citet{oncomputinghantun} 基于几何直观,发现二维流形网格的“把手”和“洞”等特征对于拓扑简化和特征检测都很有帮助。

\section{单纯同调论}

单纯同调论 (Simplicial homology) 是描述欧氏空间中具有组合结构的空间的同调论,又被称为组合拓扑学。

$ R_n $ 中处于一般位置的 $ n + 1 $ 个点 $ \{a_0, \dots, a_n\} $ ($n > 0$)的凸组合被称为一个 $ n $ 维单纯形,简称 n-单形,记作 $ (a_0, a_1, \dots, a_n) $。称 $ a_i$ 为其顶点。

设 $ K $ 为以单形为元素的有限集合,如果 $ K $ 满足
\begin{enumerate}
    \item $ K $ 中任意两个 $m$-单形交集为空集,或相交部分为某个在 K 中的 $m-1$-单形
    \item 构成 $ m$-单形顶点组成的 $ m - 1 $-单形仍在 $ K $ 中
\end{enumerate}
则称 $ K $ 为单纯复合形,又称为单复形。复形中所有维度不超过 $ m $ 的单形构成 $ K $ 的子复形,称为 $ K $ 的 m-骨架,记作 $ K^m $。

可以看出,单复形可以表示欧氏空间的一个子集,如果有这样的子集 $ X $ 可以被单复形中的所有单形的并表示,则成这样的 $ X $ 为多面体。显然,我们研究的网格就是这样的多面体。

为了构造同调群,我们还需要引入定向单形的概念。定向单形为取定了定向的单形,此时单形分为相同和相反的两种定向,这两种定向在稍后定义的链群中互为逆元。形式上的说,即考虑 m-单形 $ (p_0, p_1, \dots, p_m) $,则 
$$ p_{i_0} p_{i_1} \dots p_{i_m} = sgn(P) p_{0} p_{1} \dots p_{m} $$
其中 $ P $ 为 $ \{0, 1, 2, \dots, m \} $ 到 $ \{i_0, i_1, i_2, \dots, i_m \} $ 的一个排列,$ sgn(P) $ 为此排列的符号。

$ K $ 中所有 m-单形所对应的 m-定向单形所生成的在 $ \mathbb{Z} $ 系数域上的自由阿贝尔群称为 m-链群,记作 $ C_m(K) $。$ C_m(K) $ 的元素称为 m-链。为了叙述上的方便,定义 $ m < 0 $ 或者 $ m > \dim K $ 时 $ C_m(K) = 0 $。

定义边缘算子 $ \partial_q: T_q(K) \to C_{q-1}(K) $ 满足 
$$ \partial_q (a_0 a_1 \dots a_q) = \sum_{i=0}^{q} (-1)^i a_0 \dots \hat{a_i} \dots a_q $$

其中  $ \hat{a_i} $ 表示删除 $ a_i $,$ T_q(K) $ 为 $ K $ 的所有 q-定向单形的集合。可以看出,边缘算子描述了一个单形组合和其对应的“边界”的关系。同时可以证明,$ \partial q $ 在满足 $\partial_q(-s) = -\partial_q(s) $ 的情形下可以唯一的扩张为 $ C_q(K) \to C_{q-1}(K) $ 的同态。为了方便,此同态仍然记作 $ \partial_q $,则有
$$
\partial_q (\sum_i n_i s_i) = \sum_i n_i \partial_q (s_i)
$$
成立。

在考虑边缘同态的过程中,我们可以自然的得到 $ C_q(K) $ 的两个子群 $ Z_q(K) = ker(\partial_q(K)) $ 和 $ B_q(K) = Im(\partial_{q+1}(K)) $,分别称为 n-闭链群和 n-边缘链群。$ Z_q(K) $ 中的元素称为 n-闭链。显然,$ Z_q(K) $ 和 $ B_q(K) $ 都是自由阿贝尔群。同时可以证明,$ B_q(K) $ 是 $ Z_q(K) $ 的子群,这可以由下面的事实得到($ c \in C_q(K) $):
\begin{align*}
\partial_{q-1} (\partial_q (c)) &= \partial_{q-1} (\sum_{i=0}^q (-1)^i a_0 a_1 \dots \hat{a_i} \dots a_q) \\
&= \sum_{i=1}^q (-1)^i \partial_{q-1} (a_0 a_1 \dots \hat{a_i} \dots a_q) \\
&= \sum_{i=1}^q (-1)^i ( \sum_{j=1}^{i-1} (-1)^j a_0 \dots \hat{a_j} \dots \hat{a_i} \dots a_q + \sum_{j=i+1}^{q} (-1)^{j-1} a_0 \dots \hat{a_i} \dots \hat{a_j} \dots a_q ) \\
&= \sum_{0 \le j < i \le q} (-1)^{i+j} a_0 \dots \hat{a_j} \dots \hat{a_i} \dots a_q - \sum_{0 \le i < j \le q} (-1)^{i+j} a_0 \dots \hat{a_i} \dots \hat{a_j} \dots a_q \\
&= 0
\end{align*}

既然 $ B_q(K) $ 是 $ Z_q(K) $ 的子群,又因为 $ B_q(K) $ 是阿贝尔群,所以 $ B_q(K) $ 是正规子群。于是,我们定义q-同调群 $ H_q(K) $ 为 $ Z_q(K) / B_q(K) $,即 $ Z_q(K) $ 对 $ B_q(K) $ 的商群。如果 $ K $ 中两个闭链 $ c $,$ c' $属于同一个 $ H_q(K) $ 确定的等价类,则称这两个闭链同调,记为 $ c \sim c' $。

由于 $ Z_q(K) $ 和 $ B_q(K) $ 在 $ K $ 为有限集时均为有限生成自由阿贝尔群,则他们的商群也是有限生成阿贝尔群。由有限生成阿贝尔群的分类定理,可以将群分为自由子群和扭子群的直和:
$$
H_q(K) \simeq \underbrace{Z \oplus Z \oplus \dots \oplus Z}_{b_q(K)} \oplus Z_{i_1} \oplus \dots \oplus Z_{i_m}
$$

其中 $ b_q(K) $ 被称为 $ K $ 的 $ q $ 维 Betti 数。

我们在前面的推导过程中对链群使用的系数域为 $ \mathbb Z $。如果我们将 $ \mathbb{Z} $ 换为 $ \mathbb{Z}_2 $ 或 $ \mathbb{R} $,由于 $ \mathbb{Z}_2 $ 和 $ \mathbb{R} $ 均没有非平凡的扭子群,那么 $ H_q(K; Z_2) $ 或 $ H_q(K; R) $ 的结构就很容易得到:
$$
H_q(K;\mathbb{R}) \simeq H_q(K;\mathbb{Z}_2) \simeq\underbrace{Z \oplus Z \oplus \dots \oplus Z}_{b_q(K)}
$$

所以 $ b_q(K) $ 又可以写作 $ \dim H_q(K; \mathbb{R}) $ 。由此,我们也可以得到在 $ \mathbb{Z}_2 $ 下 $ C_q(K) $,$ Z_q(K) $,$ B_q(K) $ 和 $ H_q(K) $ 均为线性空间,有维度和基的概念。

% \section{闭曲面分类和欧拉特征数}

\section{一维链,闭链,边缘链和同调群的基表示}

下面的讨论均在 $ \mathbb{Z}_2 $ 域下进行。值得注意的是,$ \mathbb{Z}_2 $ 下 $ -1 = 1 $,所以单形的定向在此空间下只有一种,圈、路径等 $ C_m(K) $ 中的元素也不需要指明定向。

一维链群在 $ \mathbb{Z}_2 $ 下的线性空间表示很好得到,其中 $ c_i $ 为边 $ e_i $ 对应的系数。
$$
c = \begin{bmatrix}
        c_1 \\
        c_2 \\
        \cdots \\
        c_{|E|} \\
    \end{bmatrix} \in C_1(K) \quad (c_i \in \{0, 1\})
$$

$ C_1(K) $ 的(自然)基为 $ (e_1, \dots, e_{|E|}) $,其中 $ e_i $ 为 $ K $ 的 1-骨架 $ K^1 $ 的第 $ i $ 条边。

考虑 $ K^1 $ 的某个生成树 $ T $,则有 $ |V| - 1 $ 条边在生成树上, $ |E| - |V| + 1 $ 条边不在生成树上。对于不在生成树上的边 $ e_{m_i} $,生成树提供了从边的一个端点到另一个端点的路径,这个路径连同 $ e_{m_i} $ 形成一个圈,记为 $ \gamma(T, e_{m_i}) $。

这样,我们得到了 $ \{ \gamma(T, e_{m_1}), \dots, \gamma(T, e_{m_{k}}) \} $,其中 $ k = |E| - |V| + 1 $。$ \gamma(T, e_{n_i}) $ 和 $ \gamma(T, e_{n_j}) $ 间线性无关,因为 $ e_{n_j} \in \gamma(T, e_{n_i}) $ 当且仅当 $ i = j $。

对于一个任意圈 $ z $,固定树 $ T $ 上的某个顶点 $ s $,记 $ T[s, u] $ 为树上 $ s $ 到 $ u $ 的唯一路径,那么
$$
z = \sum_{e=uv \in z} e = \sum_{e=uv \in z} (T[s, u] + e + T[s, v]) = \sum_{e \in z} \gamma(T, e)
$$

其中第三个等号成立是因为所有在树上的边的 $ T[s, u] + e + T[s, v] = 0 $。由此,我们可以看出,$ \{ \gamma(T, e_{m_1}), \dots, \gamma(T, e_{m_{k}}) \} $ 构成 $ Z_1(K) $ 的一组基,矩阵表示如下:
$$
Z = 
\begin{bmatrix}
\begin{array}{c|c|c}
    \gamma(T, e_{m_1}) & \cdots & \gamma(T, e_{m_{k}})
\end{array}
\end{bmatrix}
$$

为了方便,我们引入优先基的概念。给定秩为 $ r $ 的矩阵 $ A $,由 $ A $ 的部分列向量组成的 $ B_{opt} = \{a_{i_1}, \dots, a_{i_r}\} $ 被称为优先基,如果 $ \{ i_1, \dots, i_r \} $ 是让 $ rank(B_{opt}) = rank(A) $ 的,在标号顺序上最小的一组指标集。记 $ EarliestBasis(A) $ 为 $ A $ 的优先基 $ B_{opt} $。

那么,$ B_1(K) $ 的基可以用如下表示,记为 $ B $:
$$
B = 
EarliestBasis(\begin{bmatrix}
    \begin{array}{c|c|c}
    \partial_1(f_1) & \cdots & \partial_1(f_{|F|})
\end{array}
\end{bmatrix})
$$

考虑到 $ H_1(K) = Z_1(K) / B_1(K) $,那么
$$
\begin{bmatrix}
    \begin{array}{c|c}
        B & H
    \end{array}
\end{bmatrix} =
EarliestBasis(\begin{bmatrix}
    \begin{array}{ccc|c}
        \partial_1(f_1) & \cdots & \partial_1(f_{|F|}) & Z
    \end{array}
\end{bmatrix})
$$

这是因为 $ \dim H_1 = \dim Z_1 - \dim B_1 $,而 $ \dim B_1 = rank(B) $,所以剩下的 $ rank(Z) - rank(B) $ 个列向量就是 $ H $ 的基了。

有了 $ H $ 的基,我们就可以为每个圈都计算对应的坐标,即同调向量 (homology vector) 了。

\section{计算最短同调圈}

\citet{Busaryev2012} 提出了利用标注 (annotation) 在 $ O(n^\omega + n^2 g^{\omega - 1}) $ 时间复杂度下计算 $ H_1(K; \mathbb{Z}_2) $ 的最优基的算法。

所谓最优基,就是对任意 $ c = \sum_i c_i s_i \in Z_1(K) $,定义 $ w(c) = \sum_i c_i w(s_i) $ 为闭链 $ c $ 的权,并且求一组 $ H_1(K; \mathbb{Z}_2) $ 的基使得 $ \sum_{c} w(c) $ 最小。

算法的主要思路是,在曲面上按权重从短到长排序的圈中,挑选出前 $ 2g $ 个彼此互相不同调的圈构成的基,就是最优基。为了高效的计算,这里需要解决两个问题:
\begin{enumerate}
    \item 如何减少需要考虑的圈的数量?
    \item 如何快速判断两个圈彼此是否同调?
\end{enumerate}

\subsection{利用紧性质缩小搜索空间}

问题一的解决依赖 \citet{Erickson2005} 中提出的最优基的紧性质。当一个圈 $ l $ 包含 $ l $ 上任意的两点之间的最短路径时,称 $ l $ 具有紧性质。

\begin{proposition}
    每个 $ H_1(K; \mathbb{Z}_2) $ 的最优基中的圈都具有紧性质。
\end{proposition}
\begin{proof}
    首先,最优基中的每个圈都不能分解为一些更小的圈的并。假设可以分解为两个更小的圈,那一定有一个圈可以被被其它圈表示,则去掉该圈可以找到更优的基。

    若圈 $ l_1 $ 不具有紧性质,取 $ x $ 和 $ y $ 是使得 $ l_1 $ 不包含 $ x $ 到 $ y $ 的最短路,则 $ l_1 $ 被分为两条从 $ y $ 到 $ x $ 的路,分别记为 $ \alpha $ 和 $ \beta $。记 $ \sigma $ 为 $ x $ 到 $ y $ 的最短路,则 $ l'_1 = \alpha \sigma $ 和 $ l''_1 = \beta \sigma $ 中至少有一个不能被其它圈的线性组合所表示,否则其不会出现在基中。由于 $ l'_1 $ 和 $ l''_1 $ 均比 $ l_1 $ 短,设不能被线性表示的圈为 $ l'_1 $,则我们找到了一组更优的基 $ l'_1, l_2, \dots l_{n} $。
\end{proof}

记 $ T_s $ 为从顶点 $ s $ 开始的权函数意义下的最短路径树,即包含从 $ s $ 到任意点的最短路径的树。同时,记候选圈集合为
$$
\Pi = \bigcup_{s \in V} \{ \gamma(T_s, e) | e \in E \setminus E(T_s) \}
$$
我们有如下命题成立\cite{Busaryev2012}:
\begin{proposition}
    如果按权重升序排列 $ \Pi $ 中的圈,并且为每个圈计算同调向量,那么 $ EarliestBasis(\Pi) $ 是 $ H_1(K; \mathbb{Z}_2) $ 的最优基。
\end{proposition}
\begin{proof}
    只要证明最优基中的圈一定在 $ \Pi $ 中。考虑每个圈 $ \gamma(T_s, e) $,其仅在 $ s $ 到任意一点 $ x \in \gamma(T_s, e) $ 时保留紧性质。事实上,每个紧的圈 $ l $都一定在某个 $ \gamma(T_s, e) $ 中,因为一定存在某个 $ s_i \in l $,且 $ \{\gamma(T_{s_i}, e_1), \dots, \gamma(T_{s_i}, e_{k})\} $ 生成了 1-闭链群,由于 $ l $ 的紧性质,$ l $ 一定为其中一个而不是其中某些的线性组合。因为最优基的圈一定是保留紧性质的圈,所以最优基的圈一定在 $ \Pi $ 中。
\end{proof}

通过以上命题,我们可以将候选的圈数量变为 $ O(|V|(|E|-|V|+1)) $ 。

\subsection{利用标注快速判断同调关系}

\citet{Busaryev2012} 中提出的标注方法可以帮我们解决问题二。

\begin{definition}
    一个 p-单形 的标注是一个函数 $ a: C_p(K) \to (\mathbb{Z}_2)^{g} $ 满足此性质:两个 p-圈
\end{definition}

\subsection{算法框架}

综上,我们可以得到算法的框架如下:

\begin{enumerate}
    \item 计算
\end{enumerate}