% !TeX root = ../main.tex

\chapter{基于环柄的网格切割算法}

\section{简介}

\citet{oncomputinghantun} 基于几何直观,发现二维流形网格的“把手”和“洞”等特征对于拓扑简化和特征检测都很有帮助。

\section{单纯同调论}

单纯同调论 (Simplicial homology) 是描述欧氏空间中具有组合结构的空间的同调论,又被成为组合拓扑学。

$ R_n $ 中处于一般位置的 $ n + 1 $ 个点 $ \{a_0, \dots, a_n\} $ ($n > 0$)的凸组合被称为一个 $ n $ 维单纯形,简称 n-单形,记作 $ (a_0, a_1, \dots, a_n) $。称 $ a_i$ 为其顶点。

设 $ K $ 为以单形为元素的有限集合,如果 $ K $ 满足
\begin{enumerate}
    \item $ K $ 中任意两个 $m$-单形交集为空集,或相交部分为某个在 K 中的 $m-1$-单形
    \item 构成 $ m$-单形顶点组成的 $ m - 1 $-单形仍在 $ K $ 中
\end{enumerate}
则称 $ K $ 为单纯复合形,又称为单复形。复形中所有维度不超过 $ m $ 的单形构成 $ K $ 的子复形,称为 $ K $ 的 m-骨架,记作 $ K^m $。

可以看出,单复形可以表示欧氏空间的一个子集,如果有这样的子集 $ X $ 可以被单复形中的所有单形的并表示,则成这样的 $ X $ 为多面体。显然,我们研究的网格就是这样的多面体。

为了构造同调群,我们还需要引入定向单形的概念。定向单形为取定了定向的单形,此时单形分为相同和相反的两种定向,这两种定向在稍后定义的链群中互为逆元。形式上的说,即考虑 m-单形 $ (p_0, p_1, \dots, p_m) $,则 
$$ p_{i_0} p_{i_1} \dots p_{i_m} = sgn(P) p_{0} p_{1} \dots p_{m} $$
其中 $ P $ 为 $ \{0, 1, 2, \dots, m \} $ 到 $ \{i_0, i_1, i_2, \dots, i_m \} $ 的一个排列,$ sgn(P) $ 为此排列的符号。

$ K $ 中所有 m-单形所对应的 m-定向单形所生成的在 $ \mathbb{Z} $ 系数域上的自由阿贝尔群称为 m-链群,记作 $ C_m(K) $。$ C_m(K) $ 的元素成为 m-链。为了叙述上的方便,定义 $ m < 0 $ 或者 $ m > \dim K $ 时 $ C_m(K) = 0 $。

定义边缘算子 $ \partial_q: T_q(K) \to C_{q-1}(K) $ 满足 
$$ \partial_q (a_0 a_1 \dots a_q) = \sum_{i=0}^{q} (-1)^i a_0 \dots \hat{a_i} \dots a_q $$

其中  $ \hat{a_i} $ 表示删除 $ a_i $,$ T_q(K) $ 为 $ K $ 的所有 q-定向单形的集合。可以看出,边缘算子描述了一个单形组合和其对应的“边界”的关系。同时可以证明,$ \partial q $ 在满足 $\partial_q(-s) = -\partial_q(s) $ 的情形下可以唯一的扩张为 $ C_q(K) \to C_{q-1}(K) $ 的同态。为了方便,此同态仍然记作 $ \partial_q $,则有
$$
\partial_q (\sum_i n_i s_i) = \sum_i n_i \partial_q (s_i)
$$
成立。

在考虑边缘同态的过程中,我们可以自然的得到 $ C_q(K) $ 的两个子群 $ Z_q(K) = ker(\partial_q(K)) $ 和 $ B_q(K) = Im(\partial_{q+1}(K)) $,分别成为 n-闭链群和 n-边缘链群。$ Z_q(K) $ 中的元素称为 n-闭链。显然,$ Z_q(K) $ 和 $ B_q(K) $ 都是自由阿贝尔群。同时可以证明,$ B_q(K) $ 是 $ Z_q(K) $ 的子群,这可以由下面的事实得到($ c \in C_q(K) $):
\begin{align*}
\partial_{q-1} (\partial_q (c)) &= \partial_{q-1} (\sum_{i=0}^q (-1)^i a_0 a_1 \dots \hat{a_i} \dots a_q) \\
&= \sum_{i=1}^q (-1)^i \partial_{q-1} (a_0 a_1 \dots \hat{a_i} \dots a_q) \\
&= \sum_{i=1}^q (-1)^i ( \sum_{j=1}^{i-1} (-1)^j a_0 \dots \hat{a_j} \dots \hat{a_i} \dots a_q + \sum_{j=i+1}^{q} (-1)^{j-1} a_0 \dots \hat{a_i} \dots \hat{a_j} \dots a_q ) \\
&= \sum_{0 \le j < i \le q} (-1)^{i+j} a_0 \dots \hat{a_j} \dots \hat{a_i} \dots a_q - \sum_{0 \le i < j \le q} (-1)^{i+j} a_0 \dots \hat{a_i} \dots \hat{a_j} \dots a_q \\
&= 0
\end{align*}

既然 $ B_q(K) $ 是 $ Z_q(K) $ 的子群,又因为 $ B_q(K) $ 是阿贝尔群,所以 $ B_q(K) $ 是正规子群。于是,我们定义 $ H_q(K) = Z_q(K) / B_q(K) $,即 $ Z_q(K) $ 对 $ B_q(K) $ 的商群。如果 $ K $ 中两个闭链 $ c $,$ c' $属于同一个 $ H_q(K) $ 确定的等价类,则称这两个闭链同调,记为 $ c \sim c' $。

由于 $ Z_q(K) $ 和 $ B_q(K) $ 在 $ K $ 为有限集时均为有限生成自由阿贝尔群,则他们的商群也是有限生成阿贝尔群。由有限生成阿贝尔群的分类定理,可以将群分为自由子群和扭子群的直和:
$$
H_q(K) \simeq \underbrace{Z \oplus Z \oplus \dots \oplus Z}_{b_q(K)} \oplus Z_{i_1} \oplus \dots \oplus Z_{i_m}
$$

其中 $ b_q(K) $ 被称为 $ K $ 的 $ q $ 维 Betti 数。

我们在前面的推导过程中对链群使用的系数域为 $ \mathbb Z $。如果我们将 $ \mathbb{Z} $ 换为 $ \mathbb{Z}_2 $ 或 $ \mathbb{R} $,由于 $ \mathbb{Z}_2 $ 和 $ \mathbb{R} $ 均没有非平凡的扭子群,那么 $ H_q(K; Z_2) $ 或 $ H_q(K; R) $ 的结构就很容易得到:
$$
H_q(K;\mathbb{R}) \simeq H_q(K;\mathbb{Z}_2) \simeq\underbrace{Z \oplus Z \oplus \dots \oplus Z}_{b_q(K)}
$$

所以 $ b_q(K) $ 又可以写作 $ \dim H_q(K; \mathbb{R}) $ 。由此,我们也可以得到在 $ \mathbb{Z}_2 $ 下 $ C_q(K) $,$ Z_q(K) $,$ B_q(K) $ 和 $ H_q(K) $ 均为线性空间,有维度和基的概念。

\section{计算最短同调圈}

\citet{Busaryev2012} 提出了利用标注 (annotation) 在 $ O(n^\omega + n^2 g^{\omega - 1}) $ 时间复杂度下计算 $ H_1(K; \mathbb{Z}_2) $ 的最优基的算法。

所谓最优基,就是对任意 $ c = \sum_i c_i s_i \in Z_1(K) $,定义 $ w(c) = \sum_i c_i w(s_i) $ 为闭链 $ c $ 的权,并且求一组 $ H_1(K; \mathbb{Z}_2) $ 的基使得 $ \sum_{c} w(c) $ 最小。

